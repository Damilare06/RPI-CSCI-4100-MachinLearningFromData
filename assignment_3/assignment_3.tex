\documentclass[11pt]{article}
\usepackage{enumerate}
\usepackage{fancyhdr}
\usepackage{amsmath}
\usepackage{graphicx}

\thispagestyle{empty}
\setlength{\parindent}{0cm}
\setlength{\parskip}{0.3cm plus4mm minus3mm}
\oddsidemargin = 0.0in
\textwidth = 6.5 in
\textheight = 9 in
\headsep = 0in

\title{CSCI 4100 Fall 2018 \\
% enter assignment number
Assignment 3 Answers}
\author{Damin Xu\\661679187}



\begin{document}
\maketitle
% enter question #

\noindent{\bf Exercise 1.13}
\begin{enumerate} [(a)]
	\item $P_1=\mu \lambda $\\
	$P_2=(1-\lambda)(1-\mu)$\\
	$P=P_1+P_2=\mu \lambda+(1-\lambda)(1-\mu)=1+2\mu\lambda-\mu-\lambda$
	\item Since $P=1+2\mu\lambda-\mu-\lambda$:
	\[
	P=1-\lambda+(2\lambda-1)\mu
	\]
	We want $h$ be independent of $\mu$, so $2\lambda-1 = 0$.\\
	Therefore, $\lambda=1/2$
\end{enumerate}

\noindent{\bf Exercise 2.1}
\begin{enumerate} [(1)]
\item For positive rays, break point is when k = 2.\\
Use the formula $m_H(N) = N+1$, $m_H(2)=3<2^2=4$
So it is true $m_H(k)<2^k$ at break point.

\item For positive intervals, $m_H(N)=\frac{1}{2} N^2+\frac{1}{2} N+1$.\\
Compute:
\[
\frac{1}{2} N^2+\frac{1}{2} N+1 < 2^N
\]
Get:$k=3$\\
$m_H(3)=7<8=2^3$
So it is true $m_H(k)<2^k$ at break point.

\item Because for convex sets, $m_H(N)=2^N$, so there is no break point.
\end{enumerate}
\newpage

\noindent{\bf Exercise 2.2}
\begin{enumerate} [(a)]
\item 
\noindent{\bf Theorem 2.4.} If $m_H(k)<2^k for some value k, then:$
\[
m_H(N)\leq \sum^{k-1}_{i=0}{N \choose i}
\]for all $N$. The RHS is polynomial in $N$ of degree $k-1$
\begin{enumerate}[(i)]
\item For positive rays, break point is k = 2, and $m_H(2)=3$, so
\[
\begin{aligned}
m_H(N)&=N+1\\
&\leq \sum^{k-1}_{i=0}{N \choose i}\\
&=\sum^{1}_{i=0}{N \choose i}\\
&=1+N
\end{aligned}
\]

\item For positive intervals, break point is k = 3, so
\[
\begin{aligned}
m_H(N)&=\frac{1}{2} N^2+\frac{1}{2} N+1\\
&\leq \sum^{k-1}_{i=0}{N \choose i}\\
&=\sum^{2}_{i=0}{N \choose i}\\
&=1+N+\frac{1}{2}N(N-1)\\
&=\frac{1}{2} N^2+\frac{1}{2} N+1
\end{aligned}
\]

\item For convex sets, there is no break point, but we can still assume $k=N+1$, so
\[
\begin{aligned}
m_H(N)&=2^N\\
&\leq \sum^{k-1}_{i=0}{N \choose i}\\
&= \sum^{N+1-1}_{i=0}{N \choose i}\\
&= 2^N
\end{aligned}
\]
\end{enumerate}

\item There cannot be a hyothesis set which $m_H(N)=N+2^{[N/2]}$,\\
because $N+2^{[N/2]}$ is an exponential function and $\sum^{k-1}_{i=0}{N \choose i}$ is kind of polynomials function, so there is no k achieving
\[
N+2^{[N/2]}\leq\sum^{k-1}_{i=0}{N \choose i}
\]
\end{enumerate}

\newpage

\noindent{\bf Exercise 2.3}\\
According to the formula
\[
d_{vc}=k-1
\]
\begin{enumerate} [(i)]
\item $d_{vc}=k-1=2-1=1$
\item $d_{vc}=k-1=3-1=2$
\item $d_{vc}=k-1=\infty-1=\infty$
\end{enumerate}
\ \\\\
\noindent{\bf Exercise 2.6}
\begin{enumerate} [(a)]
\item Beacuse $E_{out}(g)\leq E_{in}(g)+ \sqrt{\frac{1}{2N}ln\frac{2M}{\delta}}$, and $M=1000 N_{in\_g}=400 N_{test\_g}=200 \delta=0.05$.\\
$E_{in}(g)=\sqrt{\frac{1}{800}ln\frac{2000}{0.05}}$\\
$E_{test}(g)=\sqrt{\frac{1}{400}ln\frac{2000}{0.05}}$\\
So $E_{test}(g)\geq E_{in}(g)$.\\\\
$E_{test}(g)$ has a higher error bar.

\item If there are too many examples for testing, the test set will be smaller and the result could be worse.
\end{enumerate}

\noindent{\bf Problem 1.11}\\
Suppose there are N examples, then for supermarkets,
\[
E_{in}=\frac{1}{N}\sum^{n}_{i=1}(10\times[h(x_n)\neq1]+[h(x_n)\neq-1])
\]
For CIA,
\[
E_{in}=\frac{1}{N}([h(x_n)\neq1]+1000\times[h(x_n)\neq-1])
\]
\newpage
\noindent{\bf Problem 1.12}
\begin{enumerate} [(a)]
\item Because we need to find the stationary point, we have to compute $E_in''(h)$ first.\\
\[
\begin{aligned}
E_{in}(h)&= \sum^{N}_{n=1}(h-y_n)^2\\
E_{in}'(h)&= \sum^{N}_{n=1}[2(h-y_n)]\\
E_{in}''(h)&= \sum^{N}_{n=1}(2)=2N
\end{aligned}
\]
so $E'_{in}(h_{mean})=0$,
\[
h_{mean}=\frac{1}{N}\sum^{N}_{n=1}y_n
\]

\item \[
F(a)= \int_{-\infty}^{a}(a-x)f(x)dx+\int_{a}{\infty}(x-a)f(x)dx
\]
\[
F'(a)= \int_{-\infty}^{a}f(x)dx-\int_{a}{\infty}f(x)dx
\]
\[
F''(a)=2f(a)
\]
Because $F''(a)>0$, we want minimun from $F'(a)=0$.\\
Then $a=x_{mid}$.\\
Let $P(y=y_i)=\frac{1}{N}(i=1,2,3...N)$, then
$F(h)=\frac{1}{N}E_{in}(h)=\frac{1}{N}\sum^N_{n=1}|h-y_n|$\\
So when $h=y_{med}, E_{in}(h)$ is the minimum.

\item As $y_N$ becomes as an outlier, $h_{mean}$ becomes more and more close to $\infty$, but $h_{med}$ does not change.
\end{enumerate}
\end{document}
